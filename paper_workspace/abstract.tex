\begin{abstract}
The present study investigates the effects of dynamic learning rate schedules on the generalization capabilities of neural networks using prominent architectures such as ResNet and LSTM/Transformer. This research primarily focuses on evaluating the efficacy of adaptive learning rate strategies in optimizing model performance across various architectures. Static learning rate schedules, while common, have been shown to contribute to overfitting, highlighting a critical limitation in enhancing model generalization. Conversely, dynamic schedules demonstrate potential in mitigating this issue, thereby underscoring the significance of adopting adaptive strategies. Experiments conducted on diverse datasets, including CIFAR-10 and IMDB reviews, provide a robust evaluation framework to understand these dynamics. Our findings suggest that dynamic learning rate schedules not only enhance generalization but also adapt effectively to different architectural demands, paving the way for refined optimization techniques in neural network training. This contribution is especially pertinent for advancing methodologies that aim to achieve superior performance while maintaining model robustness. The implications of these findings advocate for a paradigm shift towards more flexible learning rate strategies in deep learning applications, potentially influencing future research and practical implementations in the field.
\end{abstract}